% !TEX root = ../main.tex

\chapter{绪论}

\section{研究背景及意义}
蛋白质是进行生命活动的基本单位,是所有细胞和组织的重要组成部分。蛋白质序列是由20多种氨基酸残基组成的大分子,一条蛋白质序列可以看作是由多个氨基酸排列组合形成的字符串,其中氨基酸的字母表 Ω =\{A,C,D,E,F,G,H,I,K,L,M,N,P,Q,R,S,T,V,W,Y\}。由于蛋白质序列中氨基酸的排列组合顺序在蛋白质折叠为三级空间结构时起着重要作用,这样的空间结构进而最终决定蛋白质的功能,因此对于蛋白质序列进行深层次的分析是一项重要且有意义的工作。

生命科学研究进入后基因组时代以来,蛋白质组学的高速发展使得蛋白质信息的信息量膨胀极为迅速。在过去的几十年里,随着蛋白质测序技术的发展,公共数据库中的蛋白质序列数量呈现爆炸式增长。相比之下,蛋白质的功能分析任务进展相对缓慢,主要是受到了生物实验成本高昂且耗费大量人力物力的限制。研究人员利用传统机器学习和深度学习领域的模型方法对蛋白质序列进行编码,可以加速对蛋白质序列功能分析的相关预测任务的研究。例如,研究人员根据大型的公共蛋白质数据库中的已知蛋白质数据开发了多种机器学习方法 \cite{min2017deep, almagro2017deeploc},他们在基于蛋白质序列的功能相关预测任务中(例如蛋白质亚细胞定位 \cite{dong2015bean, cheng2018ploc, nair2005mimicking},蛋白质结构特性预测 \cite{liu2010high, yang2009prediction},蛋白质-蛋白质相互作用预测 \cite{qi2005random, qi2006evaluation})取得了良好的效果。传统的机器学习方法在对蛋白质序列进行编码的过程中,研究人员需要利用特征工程手动在原始蛋白质数据的领域知识上提取特征,然后再部署相关的机器学习算法,这样的做法比较耗时而且依赖于强大的蛋白质组学专业知识。随着深度神经网络的兴起,传统的特征提取方法已经在很大程度上被序列编码方法即表征学习所取代。例如基于预训练的词嵌入技术 \cite{mikolov2013distributed, pennington2014glove},它通过将组成蛋白质序列的氨基酸编码成稠密的连续低维向量获得比离散特征提取表示更好的性能 \cite{asgari2015continuous, heinzinger2019modeling, hamid2019identifying}。

为了提高深度学习模型的学习性能,预训练是一种非常有效的策略。预训练这个概念最早是在计算机视觉领域提出的,取得了很好的效果。近年来,它被广泛应用于自然语言处理(NLP)的各种任务中。预训练模型通常具有较快的收敛速度和良好的泛化性能,尤其是对于训练样本有限的任务效果显著。现有的预训练模型主要是无监督模型,例如 ELMo \cite{peters1802deep} 和 BERT \cite{devlin2018bert},它们都是计算密集型的模型,因此这两种方法对于计算资源的需求很高。由于蛋白质序列共享一些共同的序列特征,预训练可以利用大规模标记的蛋白质数据集学习序列的潜在表征,并将学习到的知识表征转移到其他稀疏数据的蛋白质功能分析问题上 \cite{filipavicius2020pre, min2021pre}。基于这个想法,我们提出了一个有监督的蛋白质预训练模型,可以对蛋白质序列进行表征学习,并且我们选取蛋白质分类任务对蛋白质序列表征学习的结果进行检验。 

大规模基因组、化学和药理学数据的出现为药物发现和药物再利用提供了新的机会。对于药物分子序列的编码表征学习为计算药理学中的各种应用方向提供了基础。药物-靶标蛋白相互作用(DTI)是药物发现和药物再利用的关键步骤,因此我们选取该任务通过预测药物分子序列和靶标蛋白序列的结合亲和力验证药物分子序列和蛋白质序列的表征学习结果。

通常开发一种新药大约需要 26 亿美元 \cite{Mullard2014New},食品药品监督管理局 (FDA) 批准可能需要长达 17 年的时间 \cite{ashburn2004drug, roses2008pharmacogenetics},对于已经批准的药物寻找一种新的用途可以避免药物开发的昂贵和漫长的过程 \cite{ashburn2004drug, strittmatter2014overcoming}。在美国,罕见病被定义为影响不到 200,000 人的疾病。尽管个别罕见,但所有罕见疾病的累积影响占人口的很大比例,总的来说,估计有 7,000 种罕见疾病影响了数百万美国人。罕见疾病研究的一个主要挑战是尽管存在这种总体的健康问题,但是没有一种罕见疾病影响到足够多的人,使得罕见病的药物研发往往落后于其他更普遍的疾病进行药物开发。因此,历史上一直缺乏针对罕见病治疗的学术和药物研究,绝大多数罕见病仍然没有治疗选择。解决这种未满足的临床需求的一种方法是通过药物再利用,或者使用市场上已有的药物来治疗与开发治疗不同的疾病。这种范式在许多情况下都取得了成功,包括甲氨蝶呤和西地那非 \cite{hashkes2014methotrexate, ghofrani2006sildenafil}。药物再利用,即为现有的药物寻找一种新用途的过程,是缩短药物开发时间和降低成本的一种策略 \cite{nosengo2016new}。 为了有效地重新利用药物完成老药新用,了解哪些蛋白质被哪些药物靶向定位是有用的,高通量筛选实验 (HTS) 通常用于检查药物对其靶标的亲和力。然而,这些实验既昂贵又耗时 \cite{cohen2002protein, noble2004protein},并且详尽的搜索是不可行的,因为有数百万种类似药物的化合物 \cite{deshpande2005frequent}和数百种潜在的靶标蛋白 \cite{manning2002protein, stachel2014maximizing}。通常的做法是将大量的药物分子在潜在空间中编码进行表征学习,然后与预测模型相结合。因此,我们可以根据先前已知的药物-靶标蛋白实验建立计算模型估计新药物-靶标对的相互作用强度。

近些年,随着公共数据库中 DTI 数据的快速增加,如 ChEMBL \cite{gaulton2012chembl} 、 DrugBank \cite{wishart2008drugbank} 和 SuperTarget \cite{gunther2007supertarget} ,已经实现了对 DTI 的大规模 \textit{in silico} 识别。 目前 DTI 的计算方法主要分为三类,即基于对接、基于相似性搜索和基于特征的方法。由于深度神经网络 (DNN) 在图像和序列数据的自动特征学习方面取得了巨大成功,一些深度学习模型也被提出来预测药物和目标之间的结合亲和力。通过输入原始药物和靶标蛋白质数据,DNN 可以提取有用的信息进行预测。

尽管利用深度神经网络的模型取得了一些进展,但是仍然有很大的空间可以改进药物和靶标蛋白的特征表示编码以增强 DTI 的预测。本文我们提出了一种新方法对蛋白质序列进行嵌入表征学习并且将药物分子表示成基于原子和基于子结构的图表示来增强化合物分子的嵌入表征学习。


\section{课题研究内容}
蛋白质一共包含四级结构信息,其中蛋白质序列是蛋白质的一级结构信息。相比于蛋白质的其他结构信息,蛋白质序列的数据库更为庞大。蛋白质序列决定着蛋白质的结构,而蛋白质的结构最终决定了蛋白质的功能。由于蛋白质序列的丰富性和在其细胞生物学中的重要作用,它被认为是一类重要的蛋白质,近年来,一些研究集中在蛋白质序列及其功能分析上 \cite{dyson2005intrinsically, sugase2007mechanism, he2009predicting},因此对于蛋白质序列的表征学习研究是蛋白质上层结构研究的基础。

基于蛋白质序列表征学习的蛋白质分类任务广泛利用了自然语言处理中的文本分类技术 \cite{ganapathiraju2005computational}。受益于深度学习模型和词嵌入方法,文本分类取得了很大进展,这也为提高蛋白质分类任务的性能带来了机会。然而,将词嵌入技术从自然语言处理应用于蛋白质序列表示学习领域存在一定的困难。一方面,相比于在自然语言中可以通过单词或者词组对句子进行切分,氨基酸序列中没有特定语义的单词,因此如何切分蛋白质序列是一个需要考虑的问题。另一方面,蛋白质序列是由 20 多种氨基酸排列组合构成的,蛋白质序列的字母表要小得多,同时蛋白质序列的长度比自然语言句子要长得多,这给学习模型带来了新的挑战。

本文构建预训练-微调两阶段的有监督模型对蛋白质序列进行表征学习,同时选取蛋白质序列分类问题作为下游任务来验证蛋白质序列表征学习的结果。我们选取了三个不同数据尺度的下游蛋白质分类任务观察蛋白质序列表征学习的性能,分别是 III 型分泌效应蛋白的识别、蛋白质亚细胞定位的预测和信号肽的识别。另外,我们实现了一个网络服务,允许用户上传他们的训练和测试数据,训练数据用于对预训练模型进行微调,最终在网站上显示对测试数据的预测结果。

药物分子也具有多级结构信息,包括药物分子序列,药物化合物结构信息以及药物三维空间结构信息。其中,对于药物分子序列的编码表征学习是药物基因组学的基础。通过利用数据驱动,对药物分子序列进行特征提取完成表征学习,将药物分子序列表示成为一个稠密的低维嵌入表征向量,计算模型可以学习到药物的潜在化学信息。一个优秀的药物分子序列嵌入表征是药物再利用和药物发现等问题的基础。

然而,单独研究药物分子的表征学习是没有意义的,因此我们选取药物-靶标蛋白结合亲和力预测任务(DTI)对药物分子序列和靶标蛋白质序列的表征学习进行验证。对于 DTI 的预测是药物开发和药物再利用的关键步骤,在老药新用中发挥了至关重要的作用。新药的开发成本高、耗时长,而且往往伴随着安全问题。药物再利用可以通过为已经批准的药物寻找新用途来避免昂贵且漫长的药物开发过程。为了有效地重新利用药物,了解哪些蛋白质被哪些药物靶向定位是有用的。预测新药物-靶标对相互作用强度的计算模型可以加快药物的再利用过程,目前已经为这项任务提出了多种计算模型。但是,这些模型将药物表示简单的看作是字符串,这并不是自然地表示药物分子的方式。

本文通过引入自然语言处理中的词嵌入方法(Doc2Vec 和 GloVe)和深度学习模型(TextCNN)对靶标蛋白序列进行表征学习;将药物分子序列表示成基于原子和基于子结构的分子图结构,利用图卷积神经网络模型(GCN)对药物分子进行表征学习,达到预测药物-靶标蛋白结合亲和力预测的目的。

\section{课题研究难点}

针对蛋白质序列和药物分子序列的表征学习,研究人员已经进行了大量的研究和探索。自然语言处理中的词嵌入和预训练方法可以应用到蛋白质序列和药物分子序列的表征学习中。可以通过将序列切分成一段一段单词的有序集合,构建词向量,利用嵌入词向量最后表征整条序列的特征。然而仅仅将 NLP 中的词嵌入方法利用到生物信息序列中远远不够。

综上所述,该课题主要存在以下难点:

1. 组成蛋白质序列的氨基酸种类词库很小,同时蛋白质序列又很长,如何处理蛋白质长度分布不均的问题;

2. 蛋白质序列中没有特定语义的单词,如果对蛋白质序列进行切分训练词向量;

3. 公共数据库中的蛋白质序列数量巨大,如何利用海量的数据完成预训练;

4. 如何有效的利用蛋白质嵌入词向量来构造蛋白质分类模型验证蛋白质序列表征学习的效果;

5. 药物分子序列表示中并不仅仅含有化学元素,存在大量的无意义分割符号,如何将药物分子序列表示成有意义的结构信息;

6. 如何对药物分子图结构切割出有意义的子结构部分;

7. 药物分子序列中不同元素发挥的作用不同,如何衡量每一个元素的权重;

因此,本课题中如何有效的构建蛋白质序列的表征向量、蛋白质序列分类模型、构建药物分子序列的表征向量和 DTI 模型是课题研究的重点。

\section{本文组织结构}

本文主要介绍了蛋白质序列和药物分子序列的表征学习研究的背景,蛋白质序列表征学习的方法、基于蛋白质序列表征学习的蛋白质分类问题、药物分子序列的结构表示、药物分子序列表征学习的方法、基于蛋白质和药物分子序列表征学习的药物-靶标对结合亲和力预测问题是本文需要解决的主要问题。后续内容按照以下的结构进行组织:

第二章

第三章

第四章


