% !TEX root = ../main.tex

\chapter{蛋白质和药物分子序列表征学习研究现状}

生命科学研究进入后基因组时代以来,基因组学、蛋白质组学的高速发展使得蛋白质信息和药物化学的信息量膨胀极为迅速。在过去的几十年里,随着测序技术的发展,公共数据库中的蛋白质序列和药物分子序列的数量呈现爆炸式增长。目前在蛋白质方面已经建立了 Uniprot、Prosite、Protein Data Bank(PDB)等大型数据库用于各种蛋白质功能分析任务的研究,在药物分子方面也建立了如 ChEMBL、DrugBank、SuperTarget 等权威数据库用于各种药物再利用和药物生成等任务的研究。随着蛋白质组学和基因组学带来的数据量,研发过程中积累的知识量以及机器学习算法的快速变化,以数据驱动的机器学习和深度学习方法极大的提高了人类对蛋白质领域和药物化学领域的理解,激发了更多的研究人员投入到相关的研究中。目前,研究人员已经提出了多种针对蛋白质序列和药物分子序列的编码工具,下面主要从序列编码方法和基于编码的序列预测模型两方面对蛋白质序列和药物分子序列的研究现状进行介绍。

\section{蛋白质序列表征学习研究现状}

\subsection{蛋白质序列编码方法}
一个优秀的蛋白质序列编码方法是所有蛋白质功能分析任务,例如蛋白质分类、蛋白质亚细胞定位、蛋白质结构预测等功能分析任务的基础。针对蛋白质序列研究的关键在于如何对蛋白质序列进行编码,从而提取到序列中潜在的结构和领域知识特征。近年来,随着硬件的提升以及深度学习的不断发展,自然语言处理中的词嵌入技术替代了传统的手工特征提取方式,在序列表示学习方面取得了突破性的进展,我们可以以独立于领域知识的方式部署一种可扩展地逼近更广泛的基本蛋白质特征的方法。由于蛋白质序列和 NLP 中的文本序列存在一定的相似性,传统文本中的编码方式也被广泛应用到了蛋白质序列中。

ProtVec \cite{asgari2015continuous} 是由 Ehsaneddin 和 Mohammad 等人于 2015 年提出。他们提出了一种针对蛋白质序列的无监督数据驱动分布式表示,可以应用于生物信息学中的广泛问题,例如蛋白质可视化、蛋白质家族分类、蛋白质结构预测、蛋白质域提取等。他们使用 Swiss-Prot 大型语料库(共有 546,790 条蛋白质序列)来训练蛋白质序列的分布式表示。接下来,他们将蛋白质序列分解为子序列(即具有生物意义的单词),利用 n-gram 建模 \footnote{n-gram 表示由 n 个氨基酸组成的具有生物意义的单词} 来训练序列的通用分布式表示。与传统的 n-gram 重叠窗口不同的是,他们将一条序列分解成长度为 n 个相邻氨基酸残基滑动窗口构成的分词列表,针对不同的序列切分起始位置,一条蛋白质序列可以划分为 n 种有序非重叠单词的集合,因此得到了 $546,790 \times 3 = 1,640,370$ 条 3-gram 组成的语料库。之后,他们通过 Word2Vec \cite{mikolov2013distributed} 中的 Skip-gram 神经网络基于语料库训练 n-gram 词嵌入表示,表征序列的理化特性。在训练词向量表示时,Skip-gram 试图最大化观察到的单词序列上下文的概率,即对于给定的单词训练序列,通过找到它们对应的 $n$ 维向量,最大化其平均对数概率函数,从而单词的含义由其上下文(即相邻单词)表征。该方法的优势在于词嵌入向量只需训练一次,然后可用于编码特定问题中的蛋白质序列,无需重新训练,大大节省了资源和时间的消耗。

SeqVec \cite{heinzinger2019modeling} 是由 Heinzinger 和 Elnaggar 等人于 2019 年提出的基于深度学习的双向 ELMo \cite{peters1802deep} 模型的蛋白质序列表征学习方法。与 ProtVec 的核心思想一致,他们通过对大型未标记序列数据学习隐式的生物理化特性代替传统的显示搜索。他们在 UniRef50 数据集上训练了双向语言模型 ELMo,其中每条蛋白质序列被视为一个句子,每个氨基酸被视为一个单词。ELMo 模型在未标记的语料库上进行训练,通过给定该句子中所有先前的单词,来预测句子中最有可能出现的下一个单词。通过学习句子的概率分布,模型可以自主学习到句法和语义概念。经过训练的向量词嵌入表示是上下文化的,即给定单词的嵌入取决于其上下文。这样做的好处是两个相同的单词可以有不同的词嵌入表示,单词的词嵌入表征不是固定的,随着它们周围的单词(上下文)的不同而不同。之后他们通过将其应用到两个级别的任务来评估嵌入的预测能力,分别是每个残基(单词级别)和每个蛋白质(句子级别)的任务。结果表明,这种将无标签的语料库词嵌入迁移到下游任务中的方式是有效的,可以捕获蛋白质序列的理化特性。

UniRep \cite{alley2019unified} 是由 Ethan 等人于 2019 年提出的,他们将深度学习应用于未标记的氨基酸序列,从而将蛋白质的基本特征提炼为语义丰富、结构化和具有生物意义的基础统计表示。UniRep 模型在 2400 万个 UniRef50 一级氨基酸序列上进行了训练。该模型在一轮的训练过程中完成下一个氨基酸的预测(最小化交叉熵损失),通过不断的迭代学习如何蛋白质的内在表示。训练好的模型用于通过对中间的 mLSTM 模型的隐藏状态进行全局平均来生成输入序列的单个固定长度向量表示。在表征之上训练的顶级模型(例如,稀疏线性回归或随机森林)作为输入序列的特征化,能够对不同的蛋白质信息学任务进行监督学习。他们认为这种建立在统一表示上的嵌入表征具有广泛的适用性并且可以推广到序列空间的不可见区域。

TAPE \cite{rao2019evaluating} 是由 Roshan 和 Nicholas 等人于 2019 年提出的。由于获取监督蛋白质标签的成本高昂,他们使用半监督学习在五个与蛋白质生物学相关的领域任务上学习蛋白质的编码表征。他们采用自监督学习实现了三种架构,分别是 LSTM、Transformer 和 ResNet,使用了两种自监督损失函数(下一个单词预测和掩码预测)对模型参数进行调节。他们通过构建数据集拆分来模拟与生物学相关的泛化,例如模型泛化到序列空间中完全看不见的部分或精细解析序列空间的一小部分的能力。


\subsection{蛋白质序列分类模型}
蛋白质序列分类任务是蛋白质功能分析中一项重要的任务,很多蛋白质序列相关的分析任务都可以看作是分类任务,例如 III 型分泌效应蛋白的识别、蛋白质亚细胞定位的预测和信号肽的识别等。通常使用机器学习算法提取蛋白质序列的特征用于下游的蛋白质分类任务。在深度学习兴起之前,研究人员通常使用传统的机器学习算法,例如支持向量机(SVM)、随机森林、逻辑回归、EM 算法、贝叶斯算法等等训练蛋白质序列分类模型,能达到比较好的效果。缺点就是这些算法通常需要手动的设计特征,而特征比较依赖领域知识,对于没有深厚蛋白质生物学背景的研究人员来说不够友好。如果研究人员设计的特征没有完全反映数据的属性或者特征间存在较大的冗余,会极大地影响模型的表现结果。

随着深度学习的发展,蛋白质工程研究得到了飞速的发展。研究人员通过训练神经网络构建目标函数,利用 GPU 的强大运算能力,使得神经网络可以自动学习到蛋白质序列隐式的基础特征,有更好的普适性和扩展性。另外,通过集成学习和迁移学习可以在蛋白质工程学的相关任务上共享蛋白质的基础特征,极大的提高了模型的训练效率。

BEAN 2.0 \cite{}



\section{药物分子序列表征学习研究现状}

\subsection{药物分子序列编码方法}


\subsection{药物-靶标关系预测模型}




